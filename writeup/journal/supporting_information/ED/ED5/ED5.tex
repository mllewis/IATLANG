\documentclass[]{article}
\usepackage{lmodern}
\usepackage{amssymb,amsmath}
\usepackage{ifxetex,ifluatex}
\usepackage{tabularx}
\usepackage{supertabular}
\usepackage{lscape}
\usepackage{multirow}
\usepackage{rotating}
\usepackage{adjustbox}
\usepackage{supertabular}
\usepackage{lscape}
\usepackage{multirow}
\usepackage{tabularx}
\usepackage{csquotes}
\usepackage{upgreek}
\usepackage{hyperref}
\usepackage{lineno}
\usepackage{rotating}
\usepackage{adjustbox}
\usepackage{booktabs}

\usepackage{fixltx2e} % provides \textsubscript
\ifnum 0\ifxetex 1\fi\ifluatex 1\fi=0 % if pdftex
  \usepackage[T1]{fontenc}
  \usepackage[utf8]{inputenc}
\else % if luatex or xelatex
  \ifxetex
    \usepackage{mathspec}
  \else
    \usepackage{fontspec}
  \fi
  \defaultfontfeatures{Ligatures=TeX,Scale=MatchLowercase}
\fi
% use upquote if available, for straight quotes in verbatim environments
\IfFileExists{upquote.sty}{\usepackage{upquote}}{}
% use microtype if available
\IfFileExists{microtype.sty}{%
\usepackage{microtype}
\UseMicrotypeSet[protrusion]{basicmath} % disable protrusion for tt fonts
}{}
\usepackage[margin=1in]{geometry}
\usepackage{hyperref}
\hypersetup{unicode=true,
            pdftitle={Partial Correlations},
            pdfborder={0 0 0},
            breaklinks=true}
\urlstyle{same}  % don't use monospace font for urls
\usepackage{graphicx,grffile}
\makeatletter
\def\maxwidth{\ifdim\Gin@nat@width>\linewidth\linewidth\else\Gin@nat@width\fi}
\def\maxheight{\ifdim\Gin@nat@height>\textheight\textheight\else\Gin@nat@height\fi}
\makeatother
% Scale images if necessary, so that they will not overflow the page
% margins by default, and it is still possible to overwrite the defaults
% using explicit options in \includegraphics[width, height, ...]{}
\setkeys{Gin}{width=\maxwidth,height=\maxheight,keepaspectratio}
\IfFileExists{parskip.sty}{%
\usepackage{parskip}
}{% else
\setlength{\parindent}{0pt}
\setlength{\parskip}{6pt plus 2pt minus 1pt}
}
\setlength{\emergencystretch}{3em}  % prevent overfull lines
\providecommand{\tightlist}{%
  \setlength{\itemsep}{0pt}\setlength{\parskip}{0pt}}
\setcounter{secnumdepth}{0}
% Redefines (sub)paragraphs to behave more like sections
\ifx\paragraph\undefined\else
\let\oldparagraph\paragraph
\renewcommand{\paragraph}[1]{\oldparagraph{#1}\mbox{}}
\fi
\ifx\subparagraph\undefined\else
\let\oldsubparagraph\subparagraph
\renewcommand{\subparagraph}[1]{\oldsubparagraph{#1}\mbox{}}
\fi

%%% Use protect on footnotes to avoid problems with footnotes in titles
\let\rmarkdownfootnote\footnote%
\def\footnote{\protect\rmarkdownfootnote}

%%% Change title format to be more compact
\usepackage{titling}

% Create subtitle command for use in maketitle
\providecommand{\subtitle}[1]{
  \posttitle{
    \begin{center}\large#1\end{center}
    }
}

\setlength{\droptitle}{-2em}

  \title{Partial Correlations}
    \pretitle{\vspace{\droptitle}\centering\huge}
  \posttitle{\par}
  \subtitle{ED5}
  \author{}
    \preauthor{}\postauthor{}
      \predate{\centering\large\emph}
  \postdate{\par}
    \date{2020-05-01}


\begin{document}
\maketitle

\pagestyle{empty}
\hfill
\tiny
\begin{tabular}{lp{1.1cm}p{1.1cm}p{1.1cm}p{1.1cm}p{1.1cm}p{1.1cm}p{1.1cm}p{1.1cm}}
 & Explicit Male-Career Assoc. & Implicit Male-Career Assoc. (IAT) & Percent Women in STEM & Male-Career Assoc. (Subt.) & Male-Career Assoc. (Wiki.) & Prop. Gendered Occup. Terms & Lang. Occup. Genderness (Subt.) & Lang. Occup. Genderness (Wiki.)\\
\midrule
Explicit Male-Career Assoc. &  & $.28$ $[-.14,.62]$, 0.18 & $.16$ $[-.26,.53]$, 0.45 & $-.06$ $[-.45,.35]$, 0.78 & $.38$ $[-.03,.68]$, 0.07 & $.14$ $[-.28,.52]$, 0.51 & $.21$ $[-.21,.56]$, 0.33 & $.22$ $[-.20,.57]$, 0.31\\
\addlinespace
Implicit Male-Career Assoc.\ (IAT) & $.28$ $[-.14,.62]$, 0.18 &  & $-.38$ $[-.68,.03]$, 0.07 & $.42$ $[.02,.70]$, 0.04 & $.43$ $[.03,.71]$, 0.04 & $.48$ $[.09,.74]$, 0.02 & $.31$ $[-.11,.63]$, 0.14 & $.37$ $[-.03,.68]$, 0.07\\
\addlinespace
Percent Women in STEM & $.16$ $[-.26,.53]$, 0.45 & $-.38$ $[-.68,.03]$, 0.07 &  & $-.49$ $[-.75,-.11]$, 0.02 & $-.09$ $[-.48,.32]$, 0.67 & $-.23$ $[-.58,.19]$, 0.27 & $-.10$ $[-.48,.32]$, 0.65 & $-.46$ $[-.73,-.07]$, 0.02\\
\addlinespace
Male-Career Assoc.\ (Subt.) & $-.06$ $[-.45,.35]$, 0.78 & $.42$ $[.02,.70]$, 0.04 & $-.49$ $[-.75,-.11]$, 0.02 &  & $.47$ $[.08,.73]$, 0.02 & $.20$ $[-.23,.56]$, 0.36 & $.28$ $[-.14,.61]$, 0.18 & $.35$ $[-.07,.66]$, 0.1\\
\addlinespace
Male-Career Assoc.\ (Wiki.) & $.38$ $[-.03,.68]$, 0.07 & $.43$ $[.03,.71]$, 0.04 & $-.09$ $[-.48,.32]$, 0.67 & $.47$ $[.08,.73]$, 0.02 &  & $.11$ $[-.31,.49]$, 0.62 & $.46$ $[.06,.73]$, 0.03 & $.49$ $[.11,.75]$, 0.01\\
\addlinespace
Prop.\ Gendered Occup.\ Terms & $.14$ $[-.28,.52]$, 0.51 & $.48$ $[.09,.74]$, 0.02 & $-.23$ $[-.58,.19]$, 0.27 & $.20$ $[-.23,.56]$, 0.36 & $.11$ $[-.31,.49]$, 0.62 &  & $.53$ $[.17,.77]$, 0.01 & $.73$ $[.47,.88]$, <.001\\
\addlinespace

Lang.\ Occup.\ Genderness (Subt.) & $.21$ $[-.21,.56]$, 0.33 & $.31$ $[-.11,.63]$, 0.14 & $-.10$ $[-.48,.32]$, 0.65 & $.28$ $[-.14,.61]$, 0.18 & $.46$ $[.06,.73]$, 0.03 & $.53$ $[.17,.77]$, 0.01 &  & $.79$ $[.56,.90]$, <.001\\
\addlinespace

Lang.\ Occup.\ Genderness (Wiki.) & $.22$ $[-.20,.57]$, 0.31 & $.37$ $[-.03,.68]$, 0.07 & $-.46$ $[-.73,-.07]$, 0.02 & $.35$ $[-.07,.66]$, 0.1 & $.49$ $[.11,.75]$, 0.01 & $.73$ $[.47,.88]$, <.001 & $.79$ $[.56,.90]$, <.001 & \\
\bottomrule
\end{tabular}
\hfill
\null

\normalsize

\hypertarget{title-and-legend}{%
\subsection{Title and Legend}\label{title-and-legend}}

Pairwise Correlations partialing out the effect of median country age

Partial correlations (Pearson’s  \emph{r}) for all measures in Study 1b and 2 using language as the unit of analysis, controlling for median country age. 95\% CIs are given in brackets followed by the corresponding {\it p}-value. Implicit and explicit male-career association measures are residualized for participant age, gender, and task order. "Assoc." = association; "Lang."= language; "Subt."/ "Wiki." = Subtitle/Wikipedia corpora; "Prop. Gendered Occup. Terms." = proportion of occupation terms that are gendered. "Occup. Genderness" = degree to which occupation terms in a language tend to be associated with a particular gender in the language statistics.



\end{document}
