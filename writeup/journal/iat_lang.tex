\documentclass[man]{apa6}
\usepackage{lmodern}
\usepackage{amssymb,amsmath}
\usepackage{ifxetex,ifluatex}
\usepackage{fixltx2e} % provides \textsubscript
\ifnum 0\ifxetex 1\fi\ifluatex 1\fi=0 % if pdftex
  \usepackage[T1]{fontenc}
  \usepackage[utf8]{inputenc}
\else % if luatex or xelatex
  \ifxetex
    \usepackage{mathspec}
  \else
    \usepackage{fontspec}
  \fi
  \defaultfontfeatures{Ligatures=TeX,Scale=MatchLowercase}
\fi
% use upquote if available, for straight quotes in verbatim environments
\IfFileExists{upquote.sty}{\usepackage{upquote}}{}
% use microtype if available
\IfFileExists{microtype.sty}{%
\usepackage{microtype}
\UseMicrotypeSet[protrusion]{basicmath} % disable protrusion for tt fonts
}{}
\usepackage{hyperref}
\hypersetup{unicode=true,
            pdftitle={Language use shapes cultural norms: Large scale evidence from gender},
            pdfauthor={Molly Lewis~\& Gary Lupyan},
            pdfkeywords={cultural norms, IAT, gender},
            pdfborder={0 0 0},
            breaklinks=true}
\urlstyle{same}  % don't use monospace font for urls
\usepackage{graphicx,grffile}
\makeatletter
\def\maxwidth{\ifdim\Gin@nat@width>\linewidth\linewidth\else\Gin@nat@width\fi}
\def\maxheight{\ifdim\Gin@nat@height>\textheight\textheight\else\Gin@nat@height\fi}
\makeatother
% Scale images if necessary, so that they will not overflow the page
% margins by default, and it is still possible to overwrite the defaults
% using explicit options in \includegraphics[width, height, ...]{}
\setkeys{Gin}{width=\maxwidth,height=\maxheight,keepaspectratio}
\IfFileExists{parskip.sty}{%
\usepackage{parskip}
}{% else
\setlength{\parindent}{0pt}
\setlength{\parskip}{6pt plus 2pt minus 1pt}
}
\setlength{\emergencystretch}{3em}  % prevent overfull lines
\providecommand{\tightlist}{%
  \setlength{\itemsep}{0pt}\setlength{\parskip}{0pt}}
\setcounter{secnumdepth}{0}
% Redefines (sub)paragraphs to behave more like sections
\ifx\paragraph\undefined\else
\let\oldparagraph\paragraph
\renewcommand{\paragraph}[1]{\oldparagraph{#1}\mbox{}}
\fi
\ifx\subparagraph\undefined\else
\let\oldsubparagraph\subparagraph
\renewcommand{\subparagraph}[1]{\oldsubparagraph{#1}\mbox{}}
\fi

%%% Use protect on footnotes to avoid problems with footnotes in titles
\let\rmarkdownfootnote\footnote%
\def\footnote{\protect\rmarkdownfootnote}


  \title{Language use shapes cultural norms: Large scale evidence from gender}
    \author{Molly Lewis\textsuperscript{1,2}~\& Gary Lupyan\textsuperscript{1}}
    \date{}
  
\shorttitle{Language use shapes cultural norms}
\affiliation{
\vspace{0.5cm}
\textsuperscript{1} University of Wisconsin-Madison\\\textsuperscript{2} University of Chicago}
\keywords{cultural norms, IAT, gender\newline\indent Word count: X}
\usepackage{csquotes}
\usepackage{upgreek}
\captionsetup{font=singlespacing,justification=justified}

\usepackage{longtable}
\usepackage{lscape}
\usepackage{multirow}
\usepackage{tabularx}
\usepackage[flushleft]{threeparttable}
\usepackage{threeparttablex}

\newenvironment{lltable}{\begin{landscape}\begin{center}\begin{ThreePartTable}}{\end{ThreePartTable}\end{center}\end{landscape}}

\makeatletter
\newcommand\LastLTentrywidth{1em}
\newlength\longtablewidth
\setlength{\longtablewidth}{1in}
\newcommand{\getlongtablewidth}{\begingroup \ifcsname LT@\roman{LT@tables}\endcsname \global\longtablewidth=0pt \renewcommand{\LT@entry}[2]{\global\advance\longtablewidth by ##2\relax\gdef\LastLTentrywidth{##2}}\@nameuse{LT@\roman{LT@tables}} \fi \endgroup}


\DeclareDelayedFloatFlavor{ThreePartTable}{table}
\DeclareDelayedFloatFlavor{lltable}{table}
\DeclareDelayedFloatFlavor*{longtable}{table}
\makeatletter
\renewcommand{\efloat@iwrite}[1]{\immediate\expandafter\protected@write\csname efloat@post#1\endcsname{}}
\makeatother

\authornote{\(^*\)To whom correspondence should be addressed.
E-mail:
\href{mailto:mollylewis@uchicago.edu}{\nolinkurl{mollylewis@uchicago.edu}}

Correspondence concerning this article should be addressed to Molly
Lewis, . E-mail:
\href{mailto:mollyllewis@gmail.com}{\nolinkurl{mollyllewis@gmail.com}}}

\abstract{

}

\usepackage{amsthm}
\newtheorem{theorem}{Theorem}[section]
\newtheorem{lemma}{Lemma}[section]
\theoremstyle{definition}
\newtheorem{definition}{Definition}[section]
\newtheorem{corollary}{Corollary}[section]
\newtheorem{proposition}{Proposition}[section]
\theoremstyle{definition}
\newtheorem{example}{Example}[section]
\theoremstyle{definition}
\newtheorem{exercise}{Exercise}[section]
\theoremstyle{remark}
\newtheorem*{remark}{Remark}
\newtheorem*{solution}{Solution}
\begin{document}
\maketitle

\section{Introduction}\label{introduction}

The language we use to communicate a message shapes how our listener
interprets that message (Loftus \& Palmer, 1974; Tversky \& Kahneman,
1981; Fausey \& Boroditsky, 2010). A listener, for example, is more
likely to infer that a person is at fault if the event is described
actively (e.g., \enquote{she ignited the napkin}), as opposed to
passively (e.g., \enquote{the napkin ignited}). The formative power of
language is perhaps most potent in shaping meanings that necessarily
must be learned from others: cultural norms. In the present paper, we
consider one type of cultural norm---gender---and examine the extent to
which differences in language use may lead to cross-cultural differences
in understandings of gender.

Gender provides a useful case study of the relationship between language
and thought for several reasons. First, more abstract domains like
gender may be more subject to the influence of language relative to more
perceptually grounded domains like natural kinds (Boroditsky, 2001).
Second, many languages encode the gender of speakers and addressees
explicitly in their grammar. Third, a large body of evidence suggests
that language plays a key role in transmitting social knowledge to
children (e.g., Master, Markman, \& Dweck, 2012). And, fourth, gender
norms are highly variable across cultures and have clear and important
social implications.

For our purposes, we define the hypothesis space of possible
relationships between language and gender norms with two broad extremes:
(1) language reflects a pre-existing gender bias in its speakers
(\emph{language-as-reflection hypothesis}); (2) language causally
influences gender biases (\emph{language-as-causal hypothesis}). We
assume that the language-as-reflection hypothesis is true to some
extent: some of the ways we talk about gender reflect our knowledge and
biases acquired independently of language. For example, we may observe
that most nurses are women, and therefore be more likely to use a female
pronoun to refer to a nurse of an unknown gender. Our goal here is to
understand the extent to which language may also exert a causal
influence on conceptualizations of gender.

In particular, we explore two possible mechanisms by which the way we
speak may influence notions of
gender\footnote{These mechanisms are what Whorf (1945) refers to as phenotypes (overt) and cryptotypes (covert).}.
The first is through the overt grammatical marking of gender,
particularly on nouns, which is obligatory in roughly one quarter of
languages (e.g., in Spanish, ``nina'' (girl) and ``enfermera'' (nurse)
both take the gender marker \emph{-a} to indicate grammatical
femininity; Corbett, 1991). Because grammatical gender has a natural
link to the real world, speakers may assume that grammatical markers are
meaningful even when applied to inanimate objects that do not have a
biological sex. In addition, the mere presence of obligatory marking of
grammatical gender may promote bias by making the dimension of gender
more salient to speakers.

A second route by which language may shape gender norms is via word
co-occurrences. Words that tend to occur in similar contexts in language
may lead speakers to assume---either implicitly or explicitly---that
they have similar meanings. For example, statistically, the word
\enquote{nurse} occurs in many of the same contexts as the pronoun
\enquote{her,} providing an implicit link between these two concepts
that may lead to a bias to assume that nurses are female. This second
route may be particularly influential because the bias is encoded in
language in a way that is more implicit than grammatical markers of
gender and thus more difficult to reject.

An existing body of experimental work points to a link between language
and psychological gender bias in both adults (e.g., Phillips \&
Boroditsky, 2003) and children (e.g., Sera, Berge, \& Castillo Pintado,
1994). For example, Phillips and Boroditsky (2003) asked Spanish-English
and German-English adult bilinguals to make similarity judgements
between pairs of pictures depicting an object with a natural gender
(e.g., a bride) and one without (e.g., a toaster). They found that
participants rated pairs as more similar when the pictures matched in
grammatical gender in their native language. While these types of
studies provide suggestive evidence for a causal link between language
and psychological gender bias, they are limited by the fact that they
typically only compare speakers of 2-3 different languages and measure
bias in a way that is subject to demand characteristics.

In what follows, we ask whether the way gender is encoded linguistically
across 31 different languages predicts cross-cultural variability in a
particular manifestation of a gender bias---the bias to associate men
with careers and women with family. We begin in Study 1 by describing
cross-cultural variability in psychological gender bias using an
implicit measure. In Study 2, we use semantic-embedding models to
examine whether variability in lexical semantics predicts variability in
psychological gender biases. In Study 3, we ask whether the presence of
grammatical gender in a language is associated with greater implicit
gender bias. Together, our data suggest that both language statistics
and language structure likely play a causal role in shaping
culturally-specific notions of gender.

\section{Study 1: Gender bias across
cultures}\label{study-1-gender-bias-across-cultures}

To quantify cross-cultural gender bias, we used data from a large-scale
administration of an Implicit Association Task (IAT; Greenwald, McGhee,
\& Schwartz, 1998) by Project Implicit (Nosek, Banaji, \& Greenwald,
2002). The IAT measures the strength of respondents' implicit
associations between two pairs of concepts (e.g.,
male-career/female-family vs.~male-family/female-career) accessed via
words (e.g., \enquote{man,} \enquote{business}). The underlying
assumption of the IAT is that words denoting more similar meanings
should be easier to pair together compared to more dissimilar pairs.

Meanings are paired in the task by assigning them to the same response
keys in a 2AFC categorization task. In the critical blocks of the task,
meanings are assigned to keys in a way that is either bias-congruent
(i.e.~Key A = male/career; Key B = female/family) or bias-incongruent
(i.e.~Key A = male/family; Key B = female/career). Participants are then
presented with a word related to one of the four concepts and asked to
classify it as quickly as possible. Slower reaction times in the
bias-incongruent blocks relative to the bias-congruent blocks are
interpreted as indicating an implicit association between the
corresponding concepts (i.e.~a bias to associate male with career and
female with family).

In Study 1, we use the IAT to measure the bias to associate women with
family and men with careers across different cultures. We find a gender
bias in all countries. Replicating previous work (Miller, Eagly, \&
Linn, 2015), we also find that the magnitude of the bias is negatively
correlated with percentage of female enrollment in STEM fields.

\subsection{Methods}\label{methods}

We analyzed an existing IAT dataset collected online by Project Implicit
(\url{https://implicit.harvard.edu/implicit/}; Nosek et al., 2002). Our
analysis included all gender-career IAT scores collected from
respondents between 2005 and 2016 who had complete data and were located
in countries with more than 400 total respondents (\emph{N} = 772,467).
We further restricted our sample based on participants' reaction times
and error rates using the same criteria described in Nosek, Banjai, and
Greenwald (2002, pg.~104). Our final sample included 663,709
participants from 47 countries, with a median of 965 participants per
country. Note that although the respondents were from largely
non-English speaking countries, the IAT was conducted in English. We do
not have language background data from the participants, but we assume
that most respondents from non-English speaking countries were native
speakers of the dominant language of the country and L2 speakers of
English.

Several measures have been used in the literature to quantify the
strength of the bias from participants' responses on congruent and
incongruent blocks on the IAT. Here, we used the most robust measure,
D-score, which measures the difference between critical blocks for each
participant while controlling for individual differences in response
time (Greenwald, Nosek, \& Banaji, 2003). In addition to the implicit
measure, we also analyzed an explicit measure of gender bias. After
completing the IAT, participants were asked, \enquote{How strongly do
you associate the following with males and females?} for both the words
\enquote{career} and \enquote{family.} Participants indicated their
response on a Likert scale ranging from \emph{female} (1) to \emph{male}
(7). We calculated an explicit gender bias score for each participant as
the Career response minus the Family response, such that greater values
indicate a greater bias to associate males with career.

To obtain country-level gender bias estimates, we first calculated
residual implicit and explicit bias scores for each participant,
controling for variables that are independent predictors of bias size
(block order, participant sex, and age), and then averaged across
participants within the same country.

We compared implicit and explicit gender biases to an objective measure
gender equality that is measured for each country by the United Nations
Educational, Scientific and Cultural Organization (UNESCO): the
percentage of women among science, technology, engineering, and
mathematics (STEM) graduates in tertiary education (Miller et al., 2015;
Stoet \& Geary, 2018).

\subsection{Results}\label{results}

\begin{figure}[t]

{\centering \includegraphics{iat_lang_files/figure-latex/mapplot-1} 

}

\caption{Study 1: Residualized gender bias by country. Larger values indicate a larger bias to associate women with family and men with career.}\label{fig:mapplot}
\end{figure}

Figure 1 shows residualized implicit gender bias by country (\emph{M} =
-0.01; 0.03). Implicit gender biases were moderately correlated with
explicit gender biases both at the level of participants (\emph{r} =
0.16, \emph{p} \textless{} .0001) and countries (\emph{r} = 0.31,
\emph{p} = 0.03). we also found that implicit gender bias was negatively
correlated with percentage of women in STEM fields: Countries with a
smaller gender bias tended to have more women in STEM fields (\emph{r} =
-0.54, \emph{p} \textless{} .01). There was not a relationship between
explicit gender bias and percentage of women in STEM fields (\emph{r} =
0.09, \emph{p} = 0.63).

\subsection{Discussion}\label{discussion}

In Study 1, we replicate previously reported patterns of gender bias in
the gender-career IAT literature, with roughly comparable effect sizes
(c.f.~Nosek, et al., 2002). The weak correlation between explicit and
implicit measures is consistent with claims that these two measures tap
into different cognitive constructs (Forscher et al., 2016). In
addition, consistent with previous research (Miller et al., 2015), we
find that an objective

\section{Study 2: Gender bias and
semantics}\label{study-2-gender-bias-and-semantics}

\subsection{Study 2a: Validating embedding measure of gender
bias}\label{study-2a-validating-embedding-measure-of-gender-bias}

\subsubsection{Methods}\label{methods-1}

\subsubsection{Results}\label{results-1}

\subsubsection{Discussion}\label{discussion-1}

\subsection{Study 2b: Replication of Caliskan, et al.
(2017)}\label{study-2b-replication-of-caliskan-et-al.-2017}

\subsubsection{Methods}\label{methods-2}

\subsubsection{Results}\label{results-2}

\subsubsection{Discussion}\label{discussion-2}

\subsection{Study 2c: Cross-linguistic gender
semantics}\label{study-2c-cross-linguistic-gender-semantics}

\subsubsection{Methods}\label{methods-3}

\subsubsection{Results}\label{results-3}

\subsubsection{Discussion}\label{discussion-3}

\section{Study 3:}\label{study-3}

\subsection{Methods}\label{methods-4}

\subsection{Results}\label{results-4}

\subsection{Discussion}\label{discussion-4}

\section{General Discussion}\label{general-discussion}

\newpage

\section{References}\label{references}

\begingroup
\setlength{\parindent}{-0.5in} \setlength{\leftskip}{0.5in}

\hypertarget{refs}{}
\hypertarget{ref-boroditsky2001does}{}
Boroditsky, L. (2001). Does language shape thought?: Mandarin and
english speakers' conceptions of time. \emph{Cognitive Psychology},
\emph{43}(1), 1--22.

\hypertarget{ref-corbett1991}{}
Corbett, G. G. (1991). \emph{Gender}. Cambridge: Cambridge University
Press.

\hypertarget{ref-forscher2016meta}{}
Forscher, P. S., Lai, C., Axt, J., Ebersole, C. R., Herman, M., Devine,
P. G., \& Nosek, B. A. (2016). A meta-analysis of change in implicit
bias.

\hypertarget{ref-greenwald1998measuring}{}
Greenwald, A. G., McGhee, D. E., \& Schwartz, J. L. (1998). Measuring
individual differences in implicit cognition: The implicit association
test. \emph{Journal of Personality and Social Psychology}, \emph{74}(6),
1464.

\hypertarget{ref-greenwald2003understanding}{}
Greenwald, A. G., Nosek, B. A., \& Banaji, M. R. (2003). Understanding
and using the Implicit Association Test: An improved scoring algorithm.
\emph{Journal of Personality and Social Psychology}, \emph{85}(2), 197.

\hypertarget{ref-master2012thinking}{}
Master, A., Markman, E., \& Dweck, C. (2012). Thinking in categories or
along a continuum: Consequences for children's social judgments.
\emph{Child Development}, \emph{83}(4).

\hypertarget{ref-miller2015women}{}
Miller, D. I., Eagly, A. H., \& Linn, M. C. (2015). Women's
representation in science predicts national gender-science stereotypes:
Evidence from 66 nations. \emph{Journal of Educational Psychology},
\emph{107}(3), 631.

\hypertarget{ref-nosek2002harvesting}{}
Nosek, B. A., Banaji, M. R., \& Greenwald, A. G. (2002). Harvesting
implicit group attitudes and beliefs from a demonstration web site.
\emph{Group Dynamics: Theory, Research, and Practice}, \emph{6}(1), 101.

\hypertarget{ref-phillips2003can}{}
Phillips, W., \& Boroditsky, L. (2003). Can quirks of grammar affect the
way you think? Grammatical gender and object concepts. In
\emph{Proceedings of the 25th Annual Meeting of the Cognitive Science
Society} (pp. 928--933).

\hypertarget{ref-sera1994grammatical}{}
Sera, M. D., Berge, C. A., \& Castillo Pintado, J. del. (1994).
Grammatical and conceptual forces in the attribution of gender by
English and Spanish speakers. \emph{Cognitive Development}, \emph{9}(3),
261--292.

\hypertarget{ref-stoet2018gender}{}
Stoet, G., \& Geary, D. C. (2018). The gender-equality paradox in
science, technology, engineering, and mathematics education.
\emph{Psychological Science}, \emph{29}(4), 581--593.

\endgroup


\end{document}
